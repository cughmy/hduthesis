%!TEX root = ../main.tex

\chapter{学位论文排版元素}

通过阅读上一章,
相信您已基本配置完成编辑环境,
以及如何正确编写各级章节和段落,
了解容易引起编译出错的逃逸字符。

如果您仔细阅读过源码,
您应该已经懂得使用命令\texttt{$\backslash$begin\{xxxx\}}开始一段新的布局环境。
现在将稍系统地介绍计算机类学位论文中的排版元素,及其编写方法。
本章主要介绍以下排版元素:
\begin{itemize}
    \item 列表环境(包括有序、无序、定义三种列表)
    \item 插图和表格
    \item 代码环境
    \item 数学和算法环境
\end{itemize}

本章尽量覆盖论文写作中的大部分场景,但不面面俱到。
如有特殊需求,请仔细阅读相关宏包手册或求助于国内外TeX社区及问答网站。

\section{列表环境}

列表环境有三种,
类似与HTML的\texttt{<ol>},
\texttt{<ul>},\texttt{<dl>}
三个标签。
以下是一个定义列表环境:
\begin{description}
    \item[有序列表] enumerate 默认从阿拉伯数字1开始编号\footnote{如需更改请搜索“重定义列表”}
    \item[无序列表] itemize 默认圆点标记,尽量少用
    \item[定义列表] description 语义上用于一系列简短解释
\end{description}

列表可以嵌套,比如:
\begin{enumerate}
	\item 第一级列表
	\item 第一级列表
	\begin{enumerate}
		\item 第二级列表
		\item 第二级列表
        \begin{itemize}
            \item 第三级列表
            \item 第三级列表
		\end{itemize}
		\item 第二级列表
		\item 第二级列表
	\end{enumerate}
\end{enumerate}

\section{插图环境和浮动体}

相信您在上一章的探索学习中已经基本掌握了插入图片的方法,
但可能仍存疑虑。
现在先简单介绍浮动体的概念,
以助您理解插图环境的布局规则,
最后再介绍子图的排布以应对您更高的排版需求。
% 关于绘图,本文将在后续章节讲述
关于图的绘制,本文将在\ref{how-to-plot} 继续讲述。 % 活用ref引用,让评阅老师随处移动

当一个图片或表格太大在当前页面无法继续排版时,
一种简单的解决方案,
即是新开一页排版(Word 默认模式),
前页可能留下大段空白,十分不美观。
\LaTeX 的默认解决方案是把它们“浮动”到下一页,
与此同时将后续正文文本填充到插入点后。

插图和表格在\LaTeX 排版中默认为一个浮动体,
当排版引擎试图放置一个浮动体时,它将遵循以下规则:
\begin{enumerate}
    \item 浮动体的布局大小不得超过版心\footnote{版心是指排版文字和图表的区域,一般在页面的中心。——百度百科},否则不能通过编译(Overfull Page Error)
    \item 浮动体只能向后浮动,无法向前浮动
    \item 浮动体默认按照 h $\to$ t $\to$ b $\to$ p 规则布局
    \begin{description}
        \item[h] 排布在当前位置,如果本页所剩空间不够,忽略,检查规则 t
        \item[t] 浮动到下一页顶部
        \item[b] 浮动到下一页底部(脚注之下)
        \item[p] 浮动到一个允许出现浮动体的页面
        \item[!] 忽略浮动体放置的大多数内部参数\footnote{在下也不太懂}
    \end{description}
    \item 设置 htbp 参数的顺序不会影响默认的规则顺序
\end{enumerate}
在实践中,一般选用浮动规则[htbp], [tbp], [htp], [tp] 来完成浮动体布局。
请不要使用单一参数布局,这样极有可能出现难解的浮动问题。
不适当的浮动规则参数将导致浮动对象被放进一个队列中等待布局,
如果队列中浮动对象超过 18 个,编译时报Too Many Unprocessed Floats错误。
当需要在一页中排版的图片较多时,
您可以通过\texttt{$\backslash$clearpage}命令强制在此处排版完所有浮动体
后在排版其他内容。关于清除浮动等复杂主题,此处不再展开。

一般实践中,插图尺寸不宜超过版心一半,插图也不宜过密。
另外,可以在论文内容稳定后,
通过前置插图代码,
强行“向前浮动”,保证插图和引用处的距离不至于太远。
% 破坏语义,不宜滥用

关于本模板对浮动体的设置,参看\texttt{zjuthesis.cls},
搜索关键字“浮动体”找到对应配置。
图片引用路径在\texttt{zjuthesis.cls}里定义的\texttt{graphicspath}里,
默认情况下,\texttt{$\backslash$includegraphics}命令从论文源码根目录搜索,
如果在根目录里找到文件,则不再继续往定义引用路径搜索,
当引擎无法找到您指定的图片资源时,会导致编译错误。
注意,引用的文件名包括文件后缀。

% 现在你可以随意更动此插图代码的位置来感受一下浮动体布局的规则
\begin{figure}[htbp]
	\centering
	\begin{subfigure}[b]{.45\textwidth}  % 注意此处的尺寸控制
		\centering
		\includegraphics[width = \textwidth]{xuejian.jpg}
		\caption{仙三}\label{fig:subfig-samp1}
	\end{subfigure}
	\begin{subfigure}[b]{.45\textwidth}
		\centering
		\includegraphics[width = \textwidth]{wenhui.jpg}
		\caption{仙三外}\label{fig:subfig-samp2}
	\end{subfigure}
	\begin{subfigure}[b]{.45\textwidth}
		\centering
		\includegraphics[width = \textwidth]{lingsha.jpg}
		\caption{仙四}\label{fig:subfig-samp3}
	\end{subfigure}
	\caption{仙剑白学传}\label{fig:subfig-samp}
\end{figure}

接下来描述子图的编写,
在实际论文撰写过程中,
经常需要比较几组实验数据或场景。
此时,合乎语义的做法是为不同的组设置子图,
而不是分别设图。

多个子图组成一个单独的浮动体布局,
共用一个总图题和总引用,并可以有各自单独的子图题和引用。
本模板使用subcaption 宏包处理子图排版,如\autoref{fig:subfig-samp} 所示
论文中不可像本文一般,
平白无故地出现与行文毫无关联的图例,
而且,必须有适当的文字内容对图例做出解释。
比如,比较分析从\autoref{fig:subfig-samp1} 到\autoref{fig:subfig-samp3}
仙剑系列在白学梗方面的运用变迁。\footnote{往后数代仍有类似场景 -\_-\# (顔文字書込禁止!)}

当准备插图资源时,应该尽可能保证插图清晰,背景透明。
图中文字大小应与文中接近,不小于脚注文字大小,不大于正文段落文字大小,
框线宽度不大于2px。

如果您曾关注过图片的格式,
应该知道图片在计算机中一般分为矢量图(\autoref{fig:vector})和位图(\autoref{fig:raster})两种类型。
通俗地说,矢量图通过几何属性存储图片信息,
所以能在缩放时保持图形的几何属性。
而位图按像素点存储图片信息,在缩放时必然会丢失信息。
对于学位论文里的图例,请尽量使用矢量图,
以给评阅老师或后人精确地参考和还原实验。
常用的矢量图格式有eps, pdf, svg 和 Adobe 系列的文件格式。
其中\LaTeX 格式可以直接引用eps 和 pdf 格式的图片。

\begin{figure}[htbp]
	\centering
	\begin{subfigure}[b]{.45\textwidth}
		\centering
		\includegraphics[width = \textwidth]{vector.pdf}
		\caption{矢量图}\label{fig:vector}
	\end{subfigure}
	\begin{subfigure}[b]{.45\textwidth}
		\centering
		\includegraphics[width = \textwidth]{raster.png}
		\caption{位图}\label{fig:raster}
	\end{subfigure}
	\caption{Google Logo 的矢量图和位图比较}\label{fig:vector-raster}
\end{figure}


\section{表格}
表格与插图一样,也是浮动体单位。
在\LaTeX 中,表格的编写成本比较高,
极易引发编译错误。
初期建议同学们直接复制本模板表格进行修改。
对于只有两列的表格,建议改用列表环境完成排版。
本模板使用tabu排版表格,
使用longtabu排版超长表格。
学术论文多用线条简洁的三线表,
所谓三线就是 toprule, midrule和bottomrule 。
如\autoref{tab:tabu_test_1} 是对tabu宏包的tabu表格环境测试。
\begin{table}[htbp]
	\centering
	\caption{这是一个用tabu环境的测试用的表格}\label{tab:tabu_test_1}
    \begin{tabu}{lrr} % lrr 表示 左对齐 右对齐 右对齐
    %\begin{tabu}{|l|r|r|} % 加上竖线看看

    \toprule % 软件学院论文模板规定表头必须加粗
    \textbf{行星}     & \textbf{赤道半径}km & \textbf{公转周期}d \\
    \midrule
    水星     & 2.439  & 87.9 \\
    金星     & 6.1    & 224.682 \\
    地球     & 6378.14 & 365.24 \\
    \bottomrule
    \end{tabu}%
\end{table}

\autoref{tab:tabu_test_2} 对tabu to表格的x列模式进行测试。在表格导言区中设置为X[1]X[2]X[2],表示这三列表格的列宽比值为1:2:2,总的表格宽度由tabu to环境设置,这里设置为0.6\textbackslash linewidth。相比于tabular环境,tabu环境的列宽设置方便许多。
\begin{table}[htbp]
	\centering
	\caption{tabu环境测试表格---X列模式}\label{tab:tabu_test_2}
    \begin{tabu} to 0.6\linewidth{X[1]X[2]X[2]}
    \toprule
    \textbf{行星}     & \textbf{赤道半径}km & \textbf{公转周期}d \\  % 为了表格排版的美观 表头建议加粗
    \midrule
    水星     & 2.439  & 87.9 \\
    金星     & 6.1    & 224.682 \\
    地球     & 6378.14 & 365.24 \\
    \bottomrule
    \end{tabu}%
\end{table}

如\autoref{tab:tabu_test_3}是longtabu环境测试表格。
longtabu环境不能用在table浮动体环境中。
根据GB/T 7713.1-2006规定:如果某个表需要转页接排,
在随后的各页上应重复表的编号。
编号后跟标题(可省略)和“(续)”, % 表:「我要续…… +1
置于表上方。
续表应重复表头。

特别需要注意的是,
longtabu是基于longtable宏包开发的,
所以在zjuthesis.cls文件中已经插入了longtable宏包。
longtable环境的所有功能都可以在longtabu中使用,
如\textbackslash endhead,
\textbackslash endfirsthead,
\textbackslash endfoot,
\textbackslash endlastfoot,
和\textbackslash caption等。
具体用法请参见longtable和tabu宏包的相应文档。

\begin{longtabu}{lccc}
\caption{材料弹性模量及泊松比}\label{tab:tabu_test_3}\\
\toprule
名  称   & 弹性模量E/Gpa & 切变模量G/Gpa & 泊松比$\mu$ \\
\midrule%
\endfirsthead
\caption{材料弹性模量及泊松比(续)}\\
\toprule
名  称   & 弹性模量E/Gpa & 切变模量G/Gpa & 泊松比$\mu$ \\
\midrule%
\endhead
\bottomrule%
\endfoot
镍铬钢、合金钢 & 206    & 79.38  & 0.3 \\
碳 钢    &  196~206 & 79     & 0.3 \\
铸 钢    &  172~202 &        & 0.3 \\
球墨铸铁   &  140~154 &  73~76 & 0.3 \\
灰铸铁、白口铸铁 &  113~157 & 44     &  0.23~0.27 \\
冷拔纯铜   & 127    & 48     &   \\
轧制磷青铜  & 113    & 41     &  0.32~0.35 \\
轧制纯铜   & 108    & 39     &  0.31~0.34 \\
轧制锰青铜  & 108    & 39     & 0.35 \\
铸铝青铜   & 103    & 41     & 0.3 \\
冷拔黄铜   &  89~97 &  34~36 &  0.32~0.42 \\
轧制锌    & 82     & 31     & 0.27 \\
硬铝合金   & 70     & 26     & 0.3 \\
轧制铝    & 68     &  25~26 &  0.32~0.36 \\
铅      & 17     & 7      & 0.42 \\
玻璃     & 55     & 22     & 0.25 \\
混凝土    &  14~39 &  439~15.7 &  0.1~0.18 \\
纵纹木材   &  9.8~12 & 0.5    &   \\
横纹木材   &  0.5~0.98 &  0.44~0.64 &   \\
橡胶     & 0.00784 &        & 0.47 \\
电木     &  1.96~2.94 &  0.69~2.06 &  0.35~0.38 \\
赛璐珞    &  1.71~1.89 &  0.69~0.98 & 0.4 \\
可锻铸铁   & 152    &        &  \\
拔制铝线   & 69     &        &  \\
大理石    & 55     &        &  \\
花岗石    & 48     &        &  \\
石灰石    & 41     &        &  \\
尼龙1010 & 1.07   &        &  \\
夹布酚醛塑料 &  4~8.8 &        &  \\
石棉酚醛塑料 & 1.3    &        &  \\
高压聚乙烯  &  0.15~0.25 &        &  \\
低压聚乙烯  &  0.49~0.78 &        &  \\
聚丙烯    &  1.32~1.42 &        &  \\
硬聚氯乙烯  &  3.14~3.92 &        &  \\
聚四氟乙烯  &  1.14~1.42 &        &  \\
\end{longtabu}%


\section{代码段}

原则上,论文中应尽可能少的出现工程代码。
如果您必须引用一小段代码,
可以使用\texttt{lstlisting}设置代码环境。
本模板的代码环境默认配置在\texttt{zjuthesis.cls},
您可以搜索关键字“代码”找到配置。

本模板不鼓励引用大段代码,
所以默认情况下不为代码环境开启行号功能。
观察\autoref{code:samp-code},结合前述图表设置,
试图理解代码环境的编写。

\begin{lstlisting}[language=C++,numbers=left, numberstyle=\tiny,label=code:samp-code, caption=一段Chromium的源代码]
// Start tasks to take all the threads and block them.
  const int kNumBlockTasks = static_cast<int>(kNumWorkerThreads);
  for (int i = 0; i < kNumBlockTasks; ++i) {
    EXPECT_TRUE(pool()->PostWorkerTask(
        FROM_HERE,
        base::Bind(&TestTracker::BlockTask, tracker(), i, &blocker)));
  }
  tracker()->WaitUntilTasksBlocked(kNumWorkerThreads);

  // Setup to open the floodgates from within Shutdown().
  SetWillWaitForShutdownCallback(
      base::Bind(&TestTracker::PostBlockingTaskThenUnblockThreads,
                 scoped_refptr<TestTracker>(tracker()), pool(), &blocker,
                 kNumWorkerThreads));
  pool()->Shutdown(kNumWorkerThreads + 1);

  // Ensure that the correct number of tasks actually got run.
  tracker()->WaitUntilTasksComplete(static_cast<size_t>(kNumWorkerThreads + 1));
  tracker()->ClearCompleteSequence();
\end{lstlisting}

引用一两行代码,可以直接使用\texttt{verbatim}环境完成。
注意,此环境不会采取任何主动断行策略。
\begin{verbatim}
Error: Command failed: /bin/sh -c rsync -arvq --exclude cache
--exclude .git 
\end{verbatim}

\section{数学和算法环境}

\TeX 模板引擎创立之初就是为了更美观地排版数学公式。
在理工科的学位论文中,数学符号和数学公式必不可少
\footnote{至于定理、引理和推论等纯理科环境,本模板未作任何设定,不讨论。}。
在本模板中,数学环境由amsmath和amssymb宏包支持。
(即便没有使用公式,您应该也希望看到$a_1$, $a_2$, $C_n^m$而不是a1, a2, Cnm 吧?)

简单的行内公式,
直接在源码处编写\texttt{\$...\$}内的公式即可,
不熟习\LaTeX 公式编写的同学,
可以使用可视化的公式编辑器产生\LaTeX 代码,
这里推荐使用Daum Equation Editor完成复杂公式编辑的工作。

对于单行公式,可以使用\texttt{\$\$...\$\$}创建。
$$Y=\sum_{k=1}^n X_k$$
如果需要设定交叉引用,推荐align环境创建,如\eqref{eq:samp}所示。
\begin{align}\label{eq:samp}
    f(x) & = 2(x + 1)^{2} - 1\\                  % & 用来对齐等号
		 & = 2(x^{2} + 2x +1)-1\\
		 & = 2x^{2} + 4x + 1
\end{align}

%一个矩阵
%$$\begin{bmatrix}
%1&2&3&4\\
%5&6&7&8\\
%9&10&11&12
%\end{bmatrix}$$

计算机类的学位论文
一般少不了对研究算法的描述。
本模板选用algorithmi2e宏包排版算法环境。
详细指令使用方式参见宏包使用手册
\footnote{一般有需求排布复杂算法的同学应该有一定的科研经历}。
如\autoref{algo:duplicate2}

\begin{algorithm}
\DontPrintSemicolon
\KwIn{A sequence of integers $\langle a_1, a_2, \ldots, a_n \rangle$}
\KwOut{The index of first location with the same value as in a previous location in the sequence}
$location \gets 0$\;
$i \gets 2$\;
\While{$i \leq n \land location = 0$} {
  $j \gets 1$\;
  \While{$j < i \land location = 0$} {
    % The "l" before the If makes it so it does not expand to a second line
    \lIf{$a_i = a_j$} {
      $location \gets i$\;
    }
    \lElse{
      $j \gets j + 1$\;
    }
  }
  $i \gets i + 1$\;
}
\Return{location}\;
\caption{{\sc FindDuplicate2}}
\label{algo:duplicate2}
\end{algorithm}

\section{绘图}\label{how-to-plot}

一图胜千言,经过同学们辛苦的实验积累下的数据,
相比于冗长的文字描述,
绘图呈现的信息结构将更具可读性。
使用强大的TikZ宏包,可以绘制各式各样的图例,
比如在\ref{dirtree} 的目录结构图就是使用TikZ宏包绘制完成。
通过绘图宏包得到的是矢量图,
经过缩放后仍能精确地指导打印。
遗憾的是,
由于使用TikZ宏包绘制图例的方法艰深繁杂,
非长期钻研学术者实不可速取。

% 这里删掉了一大段TikZ宏包的使用
% 太复杂了  如果不是跟老师搞学术的话真的算了

含有大量数据的统计图,
从事数据分析工作的同学可自行使用python或R语言完成绘制,
确保输出eps或pdf格式图形,使用插图环境引入即可。

对于一般的流程图,本模板推荐使用graphviz绘图工具绘制。
相对于TikZ,graphviz已经足够适合人类掌握了。
如果坚持使用可视化工具完成此类图例的绘制,
本文推荐一个在线绘图工具\texttt{https://www.draw.io}\footnote{请保证科学上网。},
该工具可以绘制流程图、UML图、实体关系图。
另外它还支持 Dropbox 同步及输出 pdf,
通过同步论文的图片引用目录,
可以最高效的完成绘图和插图的工作。

无论用何种工具完成,
时间精力成本都不会太低。
请妥善规划您的论文撰写时间,
确保顺利毕业。

\section{关于参考文献}

硕士学位论文的参考文献,
请严格按照导师和学院规定,
注重引文质量,万不可滥引。

参考文献参照国家标准《GB/T 7714-2005: 文后参考文献著录规则》
\footnote{此标准规定的学位论文引用格式并无指定需列出是“硕士学位论文”还是“博士学位论文”},
样式文件由南京大学胡海星提供。
\begin{verbatim}
http://haixing-hu.github.io/nju-thesis/
\end{verbatim}

学校规定,参考文献采用顺序编码制,
即引文处采用序号标注,参考文献表按引文序号顺序列出。
参考文献的排版需要引入同学们自己的文献数据库,
南京大学胡海星提供了一个样例数据库,见其代码仓库内\texttt{references/test.bib}。
通过各式文献管理工具(如Zotero),您可以在论文早期工作时逐渐积累文献数据库。
通过Google学术查找一篇文献时,如\autoref{fig:gscholar} 所示,点击cite,
选择BibTeX,即可得到本文献的Bib格式的各项字段。
\begin{figure}[htbp]
    \centering
    \includegraphics[width=\textwidth]{gscholar.png}
    \caption{使用Google学术查找引文的BibTeX字段}
    \label{fig:gscholar}
\end{figure}

由Google学术提供的文献类型和字段有可能不满足胡海星前辈的设定,
注意调整。
以下是常用的文献类型:
\begin{description}
    \item[期刊]          \texttt{@article}
    \item[专著]          \texttt{@book, @inbook}
    \item[译著]          \texttt{@Book, @inBook}
    \item[会议论文集]    \texttt{@proceeding, @inproceeding}
    \item[手册]          \texttt{@manual}
    \item[网页]          \texttt{@webpage, @online}
\end{description}

\begin{itemize}
    \item 比如这是一篇中文期刊\cite{lixiaodong1999}
    \item 这是几篇英文期刊\cite{christine1998, kanamori1998}
    \item 一本中文书\cite{zh-book-1}
    \item 一本中文译著\cite{anwen1988b}
    \item 一本英文书\cite{lamport1994latex, takeuti1973}
    \item 一篇中文inproceeding\cite{nonlinear1996}
    \item 中文proceeding\cite{a2-1}
    \item 英文proceeding\cite{a2-2}
    \item 中文inproceeding\cite{aczel1998}
    \item 一篇学位论文\cite{a4-1} 
    \item 其他资料:手册\cite{ipad}报纸\cite{renminribao}网页\cite{dubash2010}
\end{itemize}

在论文中设置了一个错误或丢失的引用不会引起编译错误,
引擎会在引用标记中设一个问号。
手动编译论文的顺序一般为:
\begin{verbatim}
xelatex main
bibtex main  // 生成参考文献
xelatex main
xelatex main
\end{verbatim}
而latexmk 自动化地执行了这些步骤,所以编译时间才需要20余秒之久。







\section{本章小结}

本章划分节比较多,正式行文中请尽量避免。

传播智识,单单借助文字的力量是无力的,
即使是日常博客文章,列表、插图、表格、代码都少不了。
何况是一篇用于申请硕士学位的论文呢?

一篇学位论文集长期的科研工程实践智慧于寥寥数万字。
如何合理规划论文语义和排版元素,
让即便不熟习此领域的后人能在短时间内消化,
得以继续开物前民,
是一个值得反复求索的话题。

